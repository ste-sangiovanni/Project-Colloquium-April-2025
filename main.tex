\documentclass[10pt, aspectratio=169]{beamer}

% Theme settings
\usetheme{Copenhagen}
\usecolortheme{seahorse}
\usefonttheme{serif}

\setbeamerfont{frametitle}{size=\fontsize{14}{16}\selectfont}
% Encoding & Fonts
\usepackage[T1]{fontenc}
\usepackage[utf8]{inputenc}
\usepackage{graphicx}
\usepackage{caption}
\usepackage[inkscapeformat=png]{svg}
\usepackage{amsmath, amsfonts, amssymb}
\usepackage{multicol}
\usepackage{ragged2e}
\usepackage[absolute,overlay]{textpos}
\usepackage{changepage}
\usepackage{hyperref}
\hypersetup{hidelinks}
\usepackage{tikz}
\usetikzlibrary{positioning}
\usepackage{fontawesome5} % For social media icons
\usepackage{boldline}

% Bibliography
\usepackage[style=authoryear]{biblatex}
\addbibresource{biblio.bib}

\renewcommand*{\nameyeardelim}{\addcomma\space}
\newcommand{\customcite}[1]{\textcolor{blue}{\parencite{#1}}}
\newcommand{\customcites}[1]{\textcolor{blue}{\parencites{#1}}}
\AtEveryCite{\renewcommand{\bibsetup}{\nopagenumbers}}

% Customize bibliography format
\DeclareFieldFormat{title}{\mkbibemph{#1}} 
\DeclareFieldFormat[article]{volume}{\textbf{#1}}
\DeclareFieldFormat[article]{number}{\mkbibparens{#1}} 
\DeclareFieldFormat{pages}{#1} 
\DeclareFieldFormat{doi}{\newline\mkbibacro{DOI}: \href{https://doi.org/#1}{#1}} 
\DeclareFieldFormat{url}{\newline\mkbibacro{URL}: \href{#1}{#1}}

\renewbibmacro{in:}{}
\renewcommand*{\bibfont}{\footnotesize} % Smaller font for bibliography

% Remove hyperlinks to bibliography slide
\setbeamertemplate{bibliography item}{}
\setbeamertemplate{frametitle continuation}{}
\setbeamertemplate{headline}{}
\setbeamertemplate{footline}{}
\setbeamertemplate{bibliography item}{}

% Title Page
\title{\textbf{The Strategic Balance Between Positional and Valence Issues in Party Competition}}


\author{
    \textbf{Stefano Sangiovanni}\\[0.2cm]
    \small Supervisors: Andrea Ceron\textsuperscript{1}, Giovanna M. Invernizzi\textsuperscript{2}
}

\institute{
    \footnotesize
    \textsuperscript{1} University of Milan \\
    \textsuperscript{2} Bocconi University
}

\date{\small NASP-POLS PhD Project Colloquium \\ April 22, 2025}

% Short Title and Author for Footer
\newcommand{\shorttitle}{PhD Project Colloquium}
\newcommand{\shortauthor}{Stefano Sangiovanni}

% Remove navigation symbols
\setbeamertemplate{navigation symbols}{}

% Custom Footer
 \setbeamertemplate{footline}{
   \leavevmode%
   \hbox{%
     \begin{beamercolorbox}[wd=.5\paperwidth,ht=2.25ex,dp=1ex,center]{author in head/foot}%
       \usebeamerfont{author in head/foot}\shortauthor
     \end{beamercolorbox}%
     \begin{beamercolorbox}[wd=.5\paperwidth,ht=2.25ex,dp=1ex,center]{title in head/foot}%
       \usebeamerfont{title in head/foot}\shorttitle
     \end{beamercolorbox}}%
   \vskip0pt
 }

%Si magari metti anche lo stato di avanzamento degli altri paper (es. per parliamentary speeches, dì che stai iniziando a fare questo su questi dati e con questo metodo; per quello su campaign dataset, magari puoi accennare al data frame: periodo paesi; e ai risultati preliminari legati a ipotesi teoriche, es. come ipotizziamo osserviamo che X ha effetto su Y ma solo se Z etc.). Se devi tagliare qualcosa nella presentazione taglia le cose più ovvie, un po' come ha detto Giovanna nel commento alle slide su Mannheim (es. non tutti al project colloquium conoscono la letteratura su valence, ma comunque non devi spenderci troppo tempo, cerca di consensare quelle cose in 1 slide). Chiaramente dipende da quanto tempo avrai a disposizione.

%%% Begin slides :) I will have 10 minutes 

\begin{document}
\begin{frame}[plain]
    \centering
    \titlepage
\end{frame}

\begin{frame} %Trump in 2016
    \begin{figure}
        \centering
        \includegraphics[width=\linewidth]{images/Trump1.png}
        \label{fig:enter-label}
    \end{figure}
        \begin{figure}
        \centering
        \includegraphics[width=0.7\linewidth]{images/Trump2.jpg}
        \label{fig:enter-label}
    \end{figure}
\end{frame}

\begin{frame}{Political Scandals and Valence Theory}
    \begin{itemize}
        \item Political scandals involve \textbf{norm-breaking behavior} that violates societal norms, moral codes, or values \customcites{genovese_2010, Thompson_2013} \vspace{0.3cm}
        \item Allegations of illegal, unethical, or immoral conduct directed at politicians or institutions \customcite{Rottinghaus_2023}, they attract public scrutiny and attention \customcites{Thompson_2013, Marion_2010}\vspace{0.2cm}
        \item If scandals are perceived as \textbf{negative valence information}, then voters should negatively evaluate involved politicians \customcites{doherty2014does, Rottinghaus_2023} \vspace{0.3cm}
        \item Some studies find that scandals have negative political consequences even in polarized contexts \customcites{darr2019collision, wolsky2022scandal}, while others suggest minimal impact on politicians' careers and electoral behavior \customcites{funck2021partisanship, Lee_2023}
    \end{itemize}
\end{frame}


% \begin{frame}{Valence Theory and Political Scandals}
%     \begin{itemize}
%     %its quite established in the literature that
%     \item \textbf{Valence issues} are key factors influencing voter behaviour alongside traditional left-right distinctions and policy positions \customcite{groseclose2001model, jacoby2009public, clark2009valence} \vspace{0.3cm}
%     \item Valence dimensions refer to “character-based values” like honesty, competence, charisma, likability and unity \customcites{clark2009valence, adams2001theory} \vspace{0.3cm}
%     \item Political scandals, exposing misconduct or corruption, shape perceptions of candidate integrity and competence \customcites{doherty2014does, Rottinghaus_2023} \vspace{0.3cm}
%     \item If scandals are perceived as \textbf{negative valence information}, then voters should negatively evaluate involved politicians \customcites{doherty2014does, Rottinghaus_2023} \vspace{0.3cm}
%     \item Voters should appreciate politicians who are not involved in any scandals and should judge those that are involved
%     \end{itemize}
% \end{frame}

% \begin{frame}{Political Scandals: A Definition}
%     \begin{itemize}
%         \item Characterized by \textbf{norm-breaking behavior} deviating from societal norms, moral codes, or values \customcites{genovese_2010, Thompson_2013} \vspace{0.3cm}
%         \item Allegations of illegal, unethical, or immoral conduct directed at politicians or institutions \customcite{Rottinghaus_2023}, they attract public scrutiny and attention \customcites{Thompson_2013, Marion_2010}\vspace{0.2cm}
%         \begin{itemize}
%         \item \textbf{Financial Scandal}: personal financial gain from actions (e.g., corruption, bribery, tax evasion) 
%         \item \textbf{Personal Scandal}: immoral, illegal or unethical personal behavior (e.g., drug use, adultery, sexual allegations, cheating on CV) \vspace{0.3cm}
%         \end{itemize}
%         \item Some studies find that scandals have negative political consequences even in polarized contexts \customcites{darr2019collision, wolsky2022scandal}, while others suggest minimal impact on politicians' careers and electoral behavior \customcites{funck2021partisanship, Lee_2023}
%     \end{itemize}
% \end{frame}
 % Lee_2023
        % \item Scandals often expose negative portrayals affecting politicians' integrity and trust with constituents\vspace{0.3cm}
        % \item Different types of scandals may elicit varied reactions and meanings among voters\vspace{0.3cm}

% \begin{frame}{Literature Gaps in Political Scandal Research}
%     \begin{itemize}
%         \item Limited focus on types of scandals beyond corruption, reducing generalizability \customcite{kumlin2012scandal} \vspace{0.3cm}
%         \item Insufficient research on voters' reactions to different types of scandals \vspace{0.3cm}
%         \item Lack of systematic comparisons across various contexts (moral values are country dependent), scandal types and valence informations \customcite{kumlin2012scandal} \vspace{0.3cm}
%         \item Effects of different scandal types on electoral behavior in polarized contexts remain poorly understood \customcites{puglisi2011, darr2019collision, Rottinghaus_2023} \vspace{0.3cm} 
%     \end{itemize}
% \end{frame}

% how voters process valence-related information

%when individuals and politicians share similar values, particularly co-partisanship or ideological alignment, the consequences and the magnitude of scandals may differ. Could a high level of partisanship and ideological polarization diminish the effects of a political scandal? If so, do all types of scandals have the same consequences? 

% \begin{frame}{boh}
%     \begin{itemize}
%         \item Effects of scandals may vary across supporters of different political parties \vspace{0.3cm}
%         \item Partisan allegiance might mitigate scandal impacts
%         \item Financial scandals may affect voters differently compared to personal or ethical misconduct
%         \item Shared ideological positions between politicians and voters could influence scandal relevance
%     \end{itemize}
% \end{frame}

%%%%%% Research Questions

\begin{frame}{Research Design: Two Complementary Experiments}

\begin{block}{Main Research Question}  
\centering How does different types of \textbf{political scandals} shape voter evaluations of political candidates? 
\end{block}
\vspace{0.3cm}

\textbf{Experiment 1: Conjoint Design} \customcite{hainmueller2014}  
\begin{itemize}
    \item How do voters weigh different political scandals relative to other candidate attributes, such as party affiliation, policy positions, and positive valence?
    \item Do shared values (co-partisanship, ideological alignment) moderate the impact of political scandals on voter evaluations?
\end{itemize}
\vspace{0.2cm}
\textbf{Experiment 2: Audio-Based Survey Experiment}  
\begin{itemize}
    \item How does the tone and rhetorical delivery of a scandal accusation (calm vs. aggressive) influence voter perceptions of the accused politician?
    \item Do policy positions and ideological alignment condition the effect of scandal accusations on voter attitudes?
\end{itemize} 
\end{frame}

%%%%%%%%%%%%% Conjoint Experiment

\begin{frame}{The Conjoint Experiment}
    \begin{itemize}
        %\item Randomized conjoint design (with constrained randomization for valence attributes) \customcite{hainmueller2014} \vspace{0.2cm}
        \item Present \textbf{detailed-rich fictional scenario} where two candidates compete in an actual election \customcite{Galasso} \vspace{0.3cm}
        \item Participants will express a \textbf{preference between two politicians} with differing characteristics across various attributes \vspace{0.3cm}
        \item Each respondent completes \textbf{3 tasks}, each time choosing between \textbf{2 candidates} and indicating their preferred choice \vspace{0.3cm}
        \item \textbf{Sample:} 2,000 respondents per country (USA, UK, Italy) recruited via a survey company \vspace{0.3cm}
        \item \textbf{Power Analysis:} Our sample size allows us to detect a 0.04 effect for an attribute with 5 levels with 0.84 statistical power \customcite{lukac_stefanelli_2020}
    \end{itemize}
\end{frame}


%add several elements to provide a detailed picture of the politician, including personal background info.. maybe one scandal matters more (or less) when linked with some individual traits

% \begin{frame}{Experimental Design: Attributes and Levels}
%     \begin{itemize}
%         \item Candidate's traits: Gender, Party Affiliation \vspace{0.3cm}
%         \item Include real and relevant policy issues as attributes to immerse participants in a real-world context \customcite{morton2011electoral} \vspace{0.2cm}
% \begin{itemize}
%     \item e.g. Immigration, Green policies, Education, Taxation, EU integration \vspace{0.3cm}
% \end{itemize}
%         \item Focus on Financial and Personal scandals 
%         \customcite{Rottinghaus_2023} \vspace{0.3cm}
%         \item Use neutral categories for better interpretation of AMCEs \customcite{BansakEtAl2022} \vspace{0.3cm}
% % average marginal component effects
% \item Present candidate profiles in a "short bio" format for clarity and realism
%     \end{itemize}
% \end{frame}

\begin{frame}{Experimental Design: Profile Attributes}
\begin{itemize}
    \item \textbf{General Attributes:} Gender, Party Affiliation, Incumbency Status, Position on Immigration, Position on Economic Policies
\end{itemize}
\vspace{0.3cm}
\begin{table}[!h]
\centering
\resizebox{\textwidth}{!}{%
\begin{tabular}{|c|c|}
\hline
\textbf{Attributes}                 & \textbf{Levels}                                                                                                    \\ \hlineB{3pt} 
\textbf{Political Scandal} &  \begin{tabular}[c]{@{}c@{}}No scandal \\[2pt] Investigated for unwanted sexual conduct towards staff members \\[2pt] Falsification of credentials on curriculum vitae \\[2pt] Investigated for corruption \\[2pt] Participated in a violent anti-government protest while underage

 \end{tabular} \\ \hline
\textbf{Positive Valence} &  \begin{tabular}[c]{@{}c@{}}No positive valence \\[2pt] Had 95\% of campaign statements certified as accurate by an independent fact checker \\[2pt] Led public-private partnership preventing layoffs during local economic downturn \\[2pt] Successfully rallied party support for innovative policy agenda, turning initial 30\% backing into 90\% consensus \\[2pt] Voted with party positions on 93\% of legislative votes
 \end{tabular} \\ \hline
\end{tabular}%
}
\end{table}
\end{frame}



% we will use the Average Marginal Component Effect (AMCE; Hainmueller et al., 2014), which is the oldest and most used estimator for conjoint experiments. Yet, recent criticisms of this estimator are leading scholars to prefer marginal means (MM; see Baldassarri & De Jong, 2023; Casiraghi et al., 2023) which do not express the effects in terms of a reference category. Conveniently, both these estimators and subsequent visualizations have been implemented in the R package cjregg (Leeper, 2018). 

% Exploratory analyses regarding the interaction between profile attributes (ACIEs: Average Component Interaction Effects) and between attributes of the respondents and the profiles (subgroup analyzes) will additionally be run.

%\item  the degree to which a given value of a conjoint profile feature increases or decreases respondents’ support for the overall profile relative to a baseline, averaging across all respondents and all other profile features. 
 
% without going too much into details
% Average component interaction effect
% Average marginal components effect


\begin{frame}{Data Analysis Approach}
    \begin{itemize}
\item \textbf{AMCE:} The average effect of varying one attributes of a profile on the probability that that profile will be chosen by a respondent \customcite{BansakEtAl2022}\vspace{0.3cm} 

\item \textbf{Marginal Means:} An alternative estimator that does not rely on reference categories and is gaining preference in recent research \customcite{Casiraghi} \vspace{0.3cm} 
        \item \textbf{Exploratory Analyses:}  
        \begin{itemize}
            \item \textbf{ACIEs:} Examining how the impact of one attribute (e.g. party affiliation) depends on another (e.g. scandal) \vspace{0.2cm}  
            \item \textbf{Subgroup analyses:} Preference heterogeneity across respondent characteristics \customcite{Leeper_Hobolt_Tilley_2020} \vspace{0.2cm} 
        \end{itemize}
    \end{itemize}
\end{frame}


%%%%%%%%%%%%%%%%%%%%%% Audio Experiment :D 

\begin{frame}{Audio Experiment}
\begin{itemize}

\item Investigate how the \textbf{tone of delivery} influences the effectiveness of valence attacks \customcites{Tigue2012, Gerstle2019, Kulz2023} \vspace{0.3cm}

\item Utilize open-source multi-voice \textbf{TTS technology} to simulate realistic political debates \vspace{0.3cm}

\item \textbf{Sample:} 2,000 respondents per country (USA, UK) recruited via a survey
company \vspace{0.3cm}

\item Participants will be randomly assigned to listen to three debates or read the text version. At the end of the experiment, respondents will indicate their preferred candidate \vspace{0.3cm}
%This design tests the impact of hearing the debate versus reading the text on participants' candidate preference. 
\item \textbf{Debate Structure (Approx. 2 minutes):} \vspace{0.2cm}
\begin{itemize}
\item An anchor introduces the two politicians \vspace{0.2cm}
\item One politician attacks the other over a political scandal (negative valence) \vspace{0.2cm}
\item The second politician redirects the discussion to their own policy proposals \vspace{0.2cm}
\end{itemize}

\end{itemize}
\end{frame}


\begin{frame}{Experimental Manipulations}
\begin{table}[!h]
\centering
\resizebox{\textwidth}{!}{%
\begin{tabular}{|c|c|}
\hline

\textbf{Gender "Accused" Politician}                     & \begin{tabular}[c]{@{}c@{}}Male \\ Female\end{tabular}                                  \\ \hline
\textbf{Gender "Attacking" Politician}                     & \begin{tabular}[c]{@{}c@{}}Male \\ Female\end{tabular}                                  \\ \hline
\textbf{Tone "Attacking" Politician}                     & \begin{tabular}[c]{@{}c@{}}Calm \\ Aggressive \end{tabular}                                  \\ \hline
\textbf{Policy Topic} & \begin{tabular}[c]{@{}c@{}}Promote strict border controls (Right-wing) \\
More jobs, reduced unempl (Valence issue) \\
Financial support for low-income families (Left-wing)
\end{tabular} \\ \hline 
\textbf{Valence Attack} &  \begin{tabular}[c]{@{}c@{}}Corruption \\ Sexual Allegations
 \end{tabular} \\ \hline
\end{tabular}%
}
\end{table}
\end{frame}

\begin{frame}{How are we generating the audios?}
\begin{itemize}
\item \textbf{OS Text-To-Speech Model:} VITS \customcite{kim2021conditional}, an end-to-end speech synthesis multispeaker model trained on the CSTR-VCTK Corpus \customcite{Veaux2017CSTRVC}%, which includes 44 hours of speech data from 110 English speakers with various accents, reading approximately 400 sentences each. 
\\ \vspace{0.3cm}
\item \textbf{Pipeline 1: Pre-written Scripts + TTS}
\begin{itemize}
    \item We manually write a set of debate scripts, covering different policy topics and valence attacks
    \item A Python script processes the text with the TTS model, converting it into audio while adjusting speaker gender and voice tone
\end{itemize}
\vspace{0.3cm}
\item \textbf{Pipeline 2: LLM-Generated Debates + TTS}
\begin{itemize}
    \item An LLM generates debate scripts based on prompts specifying the policy topic and the scandal
    \item The generated text is fed into the TTS model for audio synthesis
\end{itemize}
\vspace{0.3cm}
\item \textbf{Post-processing:} we apply enhancements such as noise reduction and pitch adjustments using Librosa and Soundfile to improve realism
\end{itemize}
\end{frame}

\begin{frame}{Example: Pre-written Debate Script}
\small
\textbf{Anchorman:} Welcome to today’s debate on \textbf{economic policy.} Senator Williamson, Senator Smith, thank you for being here. \\
\vspace{0.2cm}
\textbf{Senator John Williamson:} Good morning, and thank you for the opportunity to participate.\\
\vspace{0.2cm}
\textbf{Senator Jane Smith:} Good morning, I’m glad to be here.\\
\vspace{0.2cm}
\textbf{Anchorman:} Senator Williamson, let’s start with you. What is your perspective on today’s economic challenges?\\
\vspace{0.2cm}
\textbf{Senator John Williamson:} Our priority must be \textbf{job creation and unemployment reduction.} We’ve worked on policies that aims to reduce unemployment and provide more opportunities for our citizens. Our goal should be to improve living standards and ensure long-term stability.\\
\vspace{0.2cm}
\textbf{Anchorman:} Senator Smith, do you have a response? \\
\vspace{0.2cm}
\textbf{Senator Jane Smith:} Senator Williamson talks about job creation, but how can anyone take his words seriously when he’s been investigated for \textbf{unwanted sexual conduct} towards staff members? This isn’t just a matter of policy—it’s about trust, integrity, and accountability.
\end{frame}

\begin{frame}{Conclusions and Next Steps}
\textbf{Conjoint Experiment}
\begin{itemize}
    \item Findings will show the relative weight of scandal information compared to party affiliation, policy positions, and positive valence in respondents evaluations \vspace{0.2cm}
    \item Ensure proper randomization of valence attributes while maintaining realistic candidate profiles 

\end{itemize}
\vspace{0.2cm}
\textbf{Audio Experiment}
\begin{itemize}
    \item Offering insights into how tone and framing of political scandals related attacks influence candidate perceptions and voter decision-making
    \vspace{0.2cm}
    \item Validate the emotional tone of political speech (Calm vs. Aggressive). Potential approach: use SpeechBrain \customcite{speechbrain} trained on IEMOCAP
    \vspace{0.2cm}
    \item Compare the advantages of LLM-generated vs. manually written debate scripts: which approach better captures natural political discourse while maintaining experimental validity
\end{itemize}
\end{frame}

\begin{frame}[plain]
\centering
\vspace{2cm}
\textbf{\large Thank You for Your Attention!} \\ [0.2cm] 
\texttt{stefano.sangiovanni@unimi.it} \\[3cm]

% Social Media Links with Icons
\faGithub\ \href{https://github.com/ste-sangiovanni}{ste-sangiovanni} \\[0.1cm]
\faTwitter\ @stesangio \\ 
\includegraphics[width=0.03\textwidth]{images/Bluesky_logo_(black).svg.png} @stesangio.bsky.social
\vspace{0.2cm}
\end{frame}

\begin{frame}[plain, allowframebreaks, c]
    \centering
    \vspace*{2em}
    \printbibliography
\end{frame}

\begin{frame}{Appendix 1.1 - Power Analysis - 0.05 es}
    \begin{figure}
        \centering
        \includegraphics[width=0.6\linewidth]{images/pa_3task.png}
        \label{fig:enter-label}
    \end{figure}
\end{frame}

\begin{frame}{Appendix 1.2 - Power Analysis - 0.04 es}
    \begin{figure}
        \centering
        \includegraphics[width=0.6\linewidth]{images/pa_3task_0.4effect.png}
        \label{fig:enter-label}
    \end{figure}    
\end{frame}

% \begin{frame}{Appendix 2.1 - Main Theoretical Expectations - Conjoint Experiment}
% \begin{itemize}
%     \item \textbf{H1a:} Respondents will prefer candidates with no scandals over those with scandals, regardless of other candidate attributes. \vspace{0.2cm}
%     \item \textbf{H1b:} Financial scandals will have a stronger negative impact on respondent preferences than personal scandals. \vspace{0.2cm}
%     \item \textbf{H1c:} Positive valence will mitigate the negative impact of scandals on respondent preferences, with higher positive valence reducing the severity of both financial and personal scandals. \vspace{0.2cm}
%     \item \textbf{H1d:} Respondents will be more forgiving of candidates with the same party affiliation when the candidate has a scandal. \vspace{0.2cm}
%     \item \textbf{H1e:} Respondents will prefer candidates whose policy positions align more closely with their own, even if the candidate has a scandal. \vspace{0.2cm}
% \end{itemize}
% \end{frame}

% \begin{frame}{Appendix 2.2 - Main Theoretical Expectations - Conjoint Experiment}
% \textbf{Gender and Standards of Evaluation:}
% \begin{itemize}
%     \item \textbf{H2a:} Female candidates will be held to higher standards regarding scandals than male candidates, with respondents showing a stronger negative response to scandals involving female candidates. \vspace{0.2cm}
%     \item \textbf{H2b:} Female candidates with scandals will suffer a greater decline in respondent support compared to male candidates with similar scandals. \vspace{0.2cm}
% \end{itemize}

% \textbf{Policy vs. Scandal Mitigation:}
% \begin{itemize}
%     \item \textbf{H3a:} A candidate’s policy position will be more influential than their scandal history when respondents share the same political alignment as the candidate. \vspace{0.2cm}
%     \item \textbf{H3b:} The negative impact of a scandal will be diminished if the candidate’s policy position aligns closely with the respondent's preferences, particularly for co-partisan candidates. \vspace{0.2cm}
% \end{itemize}
% \end{frame}


\begin{frame}{Appendix 2.1 - Prompt Example}
        \begin{figure}
        \centering
        \includegraphics[width=\linewidth]{images/Prompt 1.png}
        \label{fig:enter-label}
    \end{figure}  
\end{frame}

\begin{frame}{Appendix 2.2 - Prompt Example}
        \begin{figure}
        \centering
        \includegraphics[width=\linewidth]{images/prompt 2.png}
        \label{fig:enter-label}
    \end{figure}  
\end{frame}

\begin{frame}{Appendix 3.1 - Full Profile Table}
\begin{table}[!h]
\centering
\resizebox{\textwidth}{!}{%
\begin{tabular}{|c||c|}
\hline
\textbf{Attributes}                 & \textbf{Levels}                                                                                                    \\ \hlineB{2pt} 
\textbf{Gender}                     & \begin{tabular}[c]{@{}c@{}}Male \\[2pt] Female\end{tabular}                                                              \\ \hline
\textbf{Party Affiliation}          & \begin{tabular}[c]{@{}c@{}}Right-Wing \\[2pt] Left-Wing\end{tabular}                                      \\ \hline
\textbf{Incumbency Status}          & \begin{tabular}[c]{@{}c@{}}Incumbent \\[2pt] Opposition\end{tabular}                                      \\ \hlineB{2pt}  
\textbf{Position on Immigration}    & \begin{tabular}[c]{@{}c@{}}Implement strict border controls and reduce immigration \\[2pt]
Promote inclusive immigration policies and increase quotas for asylum seekers
\end{tabular}       \\ \hline
\textbf{Position on Economic Policies} & \begin{tabular}[c]{@{}c@{}}Advocates for tax reductions, market deregulation and business-friendly policies \\[2pt]
Supports stronger market regulations, higher corporate taxation and expanded welfare programs
\end{tabular} \\ \hlineB{2pt}  
\textbf{Political Scandal} &  \begin{tabular}[c]{@{}c@{}}No scandal \\[2pt] Investigated for unwanted sexual conduct towards staff members \\[2pt] Falsification of credentials on curriculum vitae \\[2pt] Investigated for corruption \\[2pt] Participated in a violent anti-government protest while underage

 \end{tabular} \\ \hline
\textbf{Positive Valence} &  \begin{tabular}[c]{@{}c@{}}No positive valence \\[2pt] Had 95\% of campaign statements certified as accurate by an independent fact checker \\[2pt] Led public-private partnership preventing layoffs during local economic downturn \\[2pt] Successfully rallied party support for innovative policy agenda, turning initial 30\% backing into 90\% consensus \\[2pt] Voted with party positions on 93\% of legislative votes
 \end{tabular} \\ \hline
\end{tabular}%
}
\end{table}
\end{frame}

\begin{frame}{Appendix 3.2 - Valence vs Valence}
    \begin{table}[!h]
\centering
\resizebox{\textwidth}{!}{%
\begin{tabular}{|c||c|}
\hline
\textbf{Attributes}                 & \textbf{Levels}                                                                                                    \\ \hlineB{2pt} 
\textbf{Political Scandal} &  \begin{tabular}[c]{@{}c@{}}No scandal \\[2pt] Investigated for unwanted sexual conduct towards staff members \\[2pt] Falsification of credentials on curriculum vitae \\[2pt] Investigated for appropriation of illegal funding \\[2pt] Participated in a violent anti-government protest while underage

 \end{tabular} \\ \hline
\textbf{Positive Valence} &  \begin{tabular}[c]{@{}c@{}}No positive valence \\[2pt] Received an award for championing workplace equity and inclusion from the National Diversity \& Inclusion Association
 \\[2pt]He had 95\% of campaign statements certified as accurate by an independent fact checker
\\[2pt] Led a public-private partnership that prevented layoffs during a local economic downturn
 \\[2pt] Received a national award for community service while underage

 \end{tabular} \\ \hline
\end{tabular}%
}
\end{table}
\end{frame}

\end{document}
